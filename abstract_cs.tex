Tato bakalářská práce se zabývá možnostmi využití evolučních algoritmů pro~generování grafických rozhraní na základě uživatelových preferencí. Tato grafická rozhraní mohou být popsána gramatikou, což je sada pravidel, která umožňuje popsat všechna jejich možná nastavení.

Dalším cílem je využítí zjištěných znalostí k vytvoření prototypu, který bude generovat grafové vizualizace nad danými datasety.

Na vytvořeném prototypu byly provedeny měření počtu iterací na~vytvoření požadované vizualizace. Jelikož jsou interaktivní evoluční algoritmy závislé na náhodě a~na uživatelových preferencích, nelze jasně říci, že by počet dosažených iterací měl vyšší vypovídající hodnotu. Tudíž měření slouží spíše jako proof of~concept.
