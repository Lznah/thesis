\chapter{Examples of usage}
To shown how the prototype works, I have chosen two usecases. Both of them defines visualizations, that the prototype should produce in the end. I also show which visualization were chosen to be bred.
\section{Usecase 1: Students performance in Exams}
The dataset contains information about students, their gender, race/ethnicity, scores in math, reading and writing and other information. It can be downloaded from Kaggle \cite{students-performance-in-exams}.

\subsection{Goal}
At first, I generated the visualization, that is shown in figure \ref{pic:students-performance-in-exams-goal}. I have chosen this visualization, because I would not be able to show crossover on an easier one. The goal of this usecase is to generate at least a similar visualization. It means, that the axes should be mapped in the same way to the fields \textbf{math score} and \textbf{reading score}, but mark properties are arbitrary. The final visualization must distinquish categories of nominal field \textbf{gender}.

\imagefigure{students-performance-in-exams-goal.png}{The desired visualization for the usecase 1.}

\subsection{Evolution}

As it is shown in figure \ref{pic:evolution-diagram-usecase1.png}, there are two individuals selected for breeding in the first iteration, which is an initialization. Thwy are chosen, because the first one contains binning of \textbf{math score} field and the second is selected, because I prefer its mark encoding, which is \textbf{fill} instead of \textbf{bar}.
Probability of change is set to zero, because I want get in the next generation only siblings, that are simple recombinations of the mappings of these two visualizations.

In the second interation, I get \textbf{math score} field mapped to \textbf{y} axis and \textbf{gender} field to \textbf{fill} mark property. The only feature, that is missing in the visualization, is the aggregation of \textbf{reading} field. This is fixed in the sixth iteration, after the individual from the second iteration is repeteadly chosen into four iterations with 15 \% probability of change. As it is alone in the population, the only genetic operator applied on these on the individual is the mutation.

In the sixth iteration, algorithm returned desired visualization. The axis are swaped and it uses different mark property, but it shows the same relation between \textbf{average reading score}, \textbf{gender} and \textbf{math score}, that is also binned. The final visualization specification is shown in listing \ref{code:specification-usecase1}.

\imagefigurefull{evolution-diagram-usecase1.png}{The diagram of chosen individuals during iterations for usecase 1.}{1.1}

\begin{listing}[htbp]
\caption{\label{code:specification-usecase1}The final visualization's specification from usecase 1.}
\begin{minted}[bgcolor=codebg]{js}
{
  /* Top-level Specifications and common properties were removed,
   because they are not evolved by the prototype. */
  "mark": "square",
  "encoding": {
    "fill": {
      "field": "gender",
      "type": "nominal"
    },
    "x": {
      "field": "reading score",
      "type": "quantitative",
      "aggregate": "average"
    },
    "y": {
      "type": "quantitative",
      "bin": true,
      "field": "math score"
    }
  }
}
\end{minted}
\end{listing}

\clearpage

\section{Usecase 2: Heart Diseases}
\subsection{Goal}
The goal for the second use case is to find such visualization of \textbf{Heart Diseases} dataset, that shows different types of chest pain \textbf{cp} by \textbf{age} and with maximum heart rate achieved \textbf{thalach}.
\subsection{Evolution}

\imagefigurelarge{evolution-diagram-usecase2.png}{The diagram of chosen individuals during iterations for usecase 2.}

\begin{listing}[htbp]
\caption{\label{code:specification-usecase2}The final visualization's specification from usecase 2.}
\begin{minted}[bgcolor=codebg]{js}
{
  /* Top-level Specifications and common properties were removed,
   because they are not evolved by the prototype. */
  "mark": "bar",
  "encoding": {
    "color": {
      "field": "cp",
      "type": "nominal"
    },
    "x": {
      "aggregate": "average",
      "field": "thalach",
      "type": "quantitative"
    },
    "y": {
      "field": "age",
      "type": "quantitative",
      "bin": true
    }
  }
}
\end{minted}
\end{listing}
