\chapter{State of the art}

Unlike traditional evolutionary computation (EC), which is widely studied and is used in various applications, interactive evolutionary computation (IEC) do not typically get the same attention, because of its non-mathematical based evaluation and its application is mostly used for art, music and design, which are not very practical uses if people can do it better by themselves. For example, well-known is a project Picbreeder \cite{picbreeder} from 2006, which is collaborative art application that allows pictures be bred like animals and shared these pictures with community, so everyone can continue with breeding their own pictures. Another interesting application of IEC for art creation is Eletric Sheep \cite{electricsheep}, that creates beautiful psychedelic visualizations that could be used as a screensaver. The evaluation is based on community rankings of newly created visualizations in a single iteration. Electric Sheep project runs since 1999 \cite{electricsheep-founded} and in its archive of created visualization \cite{electricsheep-archive}, an obvious progress can be seen after almost 250 iterations. Both examples uses Interactive Genetic Programming to map mathematical representations of images as tree-like structures.

As it was said before, IEC is also used for music. For example, GenJam \cite{genjam} project, that is developed since 1993 and that can play jazz alongside a jazz musician.

Another use case of IEC, that is not described in the paper above is a generation of GUI from a paper \textbf{Interactive Genetic Algorithms for User Interface Design} \cite{igagui}.

Of course, there are other use cases of IEC, for example industrial design, face image generation or database retrieval and they are well summarized in Hideyuki Tagaki's paper \textbf{Interactive Evolutionary Computation: Fusion of the Capabilities of EC Optimization and Human Evaluation} \cite{tagakipaper}.

IEC are not so different from traditional EC. Additionaly, the structure of the visualizations has a format of a a tree, so traditional genetic programming (GP), that was introduced in 1990 by John R. Koza in an article \textbf{Genetic Programming: A Paradigm for Genetically Breeding Populations of Computer Programs to Solve Problems} \cite{genetic-programming} is apparently a good example of an inspiration, because GP is used exactly for evolving tree-like structures. Throughout the years new ways of tree-like structure evolution were discovered. For example relatively new Grammatical Evolution (GE) \cite{grammatical-evolution} by Michael O'Neill and Connor Ryan from 2001, that is based on Koza-style GP, but allows user to define grammar. The grammar provides control over the creation of feasible solutions in such way, that every rule of the tree structure can not be expanded into arbitrary rule from a set of available rules, but has to follow given grammar. Traditionally Koza-style GP is still able to use, but it has to be bended to support this functionality.

User fatigue is a big problem for all IEC systems and using only an interactive evolutionary algorithm (IEA) might not be efficient enought in seaching desired solutions in a state space. In term of user fatique, IEA has to search as much efficiently as possible. A normal IEC process lasts 10--20 iterations, before user gets bored \cite{tagakipaper}. It is necessary to implement mechanisms to prevent evolutionary algorithm search in evidently wrong regions. This could be accomplished with additional information provided by user \cite{user-fatique-in-iec}.
