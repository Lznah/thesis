\label{Introduction}
Evolutionary process, which is one of the pillars of evolutionary biology, has proven, over almost five billions of years, to be an efficient approach to get complex and successful individuals in various environments with enormous diversity of individuals and with just a small changes of heritable characteristic over successive generations \cite{on-the-origin-of-species}.

Many methods in machine learning and artificial intelligence are actually inspired by processes seen in nature \cite{introduction-to-evolutionary-computing} and as it is evolutionary algorithms are one of them. More specifically, evolutionary algorithms emulate mechanisms of evolutionary process such as gene mutation and allelic combination (genotype). Traditionally evolutionary algorithms are usually used for finding a suitable solution (not necessarily the best solution) for difficult optimization problems, which are not possible to be solved in polynomial time, but quality of the solutions is easily numerically expressible so searching over space of feasible solutions is completely automated. In addition, these problems that satisfy the requirement are based on mathematical evaluation of their solutions, which means the evaluation is based on an objective approach. However, quality of solutions of some problems are hard to be described mathematically e.q. music and art, because quality of their representative solutions is based on subjective preferences of an observer. Not only a human-based evaluation is needed for the problems described above, but it is also way more efficient, and, besides, it can be used in combination with traditional mathematically described evaluation for very complex tasks to help directing the space search.

The aims of this thesis is to explore the field of existing methods of interactive evolutionary computation used on graphical interfaces and use given knowledge to create a prototype for generating visualizations, that are described in JSON format, which follows \textbf{Vega-Lite} \cite{vega} grammar. The implemented prototype should help data scientist to plot data without requirement of having knowledge, how to write these JSONs.
